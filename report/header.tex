\RequirePackage{ifthen}

\newcommand \type{Fachprojekt Report}
\newcommand \Autor{Lukas B{\"u}nger \\ Thilaksan Kodeeswaran\\ Dilsan Mahadeva}
\newcommand \submissiondate{30.09.2020}
\newcommand \thesistitle{Interactive Real-Time Gaming}
\newcommand \firstsupervisor{Prof.~Dr. Jian-Jia Chen}
\newcommand \secondsupervisor{M.Sc. Junjie Shi}
\newcommand \ErstLehrstuhl{Lehrstuhl Informatik 12 (Eingebettete Systeme) \\ \url{http://ls12-www.cs.tu-dortmund.de}}
\newcommand \ZweitLehrstuhl{}

\RequirePackage{ifpdf} \ifpdf
  \pdfoutput=1
  \pdftrue
  \message{pdfLaTeX}
  \documentclass[pdftex,11pt,a4paper,oneside,ngerman]{scrbook}
  \usepackage[ngerman]{babel}
  \usepackage{float}
  \usepackage[pdftex]{thumbpdf}
  \usepackage[pdftex]{graphicx}
  \usepackage[pdftex]{hyperref}
  \usepackage{pdfpages}
  \pdfoutput=1
  \pdfcompresslevel=9
  \DeclareGraphicsExtensions{.pdf,.jpg,.png}
\else
  \pdffalse
  \message{LaTeX}
  \documentclass[dvips,11pt,a4paper,oneside,ngerman]{scrbook}
  \usepackage{float}
  \usepackage{graphicx}
  \usepackage{epsf}
  \usepackage[dvips]{hyperref}
  \DeclareGraphicsExtensions{.eps}
\fi

% Informationen fuer pdf-File festlegen
\hypersetup
{
    pdfauthor = {\Autor},
    pdftitle = {\thesistitle},
    pdfsubject = {\type, TU Dortmund, Fakult{\"a}t f{\"u}r Informatik},
    pdfproducer = {LaTeX},
    pdfview = FitV,
    pdfstartview = FitV,
    pdfhighlight = /I,
    pdfborder = 0 0 0,
    colorlinks = false,
    bookmarksopen,
    bookmarksopenlevel = 1,
    bookmarksnumbered = false,
    plainpages = false
}%

% -------------------------------------------------------------------
% Seitenformat anpassen
\usepackage[a4paper,left=3.5cm,right=2.5cm,bottom=3.5cm,top=3cm]{geometry}
\setlength{\headheight}{15pt}

% -------------------------------------------------------------------
% Grafikpakete einbinden
\usepackage{amsmath,amssymb}
\usepackage{flafter}
\usepackage{subfigure}

% -------------------------------------------------------------------
\usepackage{ifthen}

% -------------------------------------------------------------------
\usepackage[absolute,overlay]{textpos}
\setlength{\TPHorizModule}{1mm}
\setlength{\TPVertModule}{\TPHorizModule}
\textblockorigin{0mm}{0mm}
\usepackage{fix-cm}
\usepackage{setspace}
\usepackage{scrhack}

% -------------------------------------------------------------------
% Korrekte Darstellung der Umlaute
% \usepackage[german,ngerman]{babel}
% \usepackage[utf8]{inputenc}
% \usepackage[T1]{fontenc}
% \usepackage{ae,aecompl}

% -------------------------------------------------------------------
% Bibtex deutsch
%\usepackage[numbers,sort,square]{natbib}
%\usepackage{bibgerm}

% -------------------------------------------------------------------
% Anführungszeichen
\usepackage[babel,german=quotes]{csquotes}

% -------------------------------------------------------------------
% URLs
\usepackage{url}

% -------------------------------------------------------------------
% Caption anpassen
\usepackage[margin=0pt,font=small,labelfont=bf]{caption}

% -------------------------------------------------------------------
% Erweitere Tabellen
\usepackage{booktabs}

% -------------------------------------------------------------------
% Eurosymbol
\usepackage{eurosym}

% -------------------------------------------------------------------
% Zeilenabstand einstellen
\renewcommand{\baselinestretch}{1.25}
% Floating-Umgebungen anpassen
\renewcommand{\topfraction}{0.9}
\renewcommand{\bottomfraction}{0.8}

% -------------------------------------------------------------------
% Keine einzelnen Zeilen beim Anfang eines Abschnitts ("Schusterjungen")
\clubpenalty = 10000
% Keine einzelnen Zeilen am Ende eines Abschnitts ("Hurenkinder")
\widowpenalty = 10000 \displaywidowpenalty = 10000
\parindent=0cm

% -------------------------------------------------------------------
% Kopfzeile hinzufuegen
\usepackage{fancyhdr}
\usepackage{extramarks}

\pagestyle{fancy}
\renewcommand{\chaptermark}[1]{\markboth{#1}{}}
\renewcommand{\sectionmark}[1]{\markright{#1}{}}

\fancyhf{}
\fancyhead[LE,RO]{\thepage}
\fancyhead[RE]{\textit{\nouppercase{\leftmark}}}
\fancyhead[LO]{\textit{\nouppercase{\rightmark}}}

\fancypagestyle{plain}{ %
\fancyhf{} % remove everything
\renewcommand{\headrulewidth}{0pt} % remove lines as well
\renewcommand{\footrulewidth}{0pt}} \pagestyle{headings}

% -------------------------------------------------------------------
% Eigene Farben definieren
\usepackage{color}
\definecolor{TUGreen}{rgb}{0.517,0.721,0.094}
\definecolor{TUOrange}{rgb}{1.0,0.7176,0.0}
\definecolor{BrightGray}{gray}{0.9}
\definecolor{DarkGray}{gray}{0.2}
\definecolor{white}{rgb}{1,1,1}
\definecolor{black}{rgb}{0,0,0}
\definecolor{red}{rgb}{1,0,0}

% -------------------------------------------------------------------
% Programm-Listings einbinden und formatieren
\usepackage{listings}

\lstdefinestyle{C++}
{
language=C++,
backgroundcolor=\color{BrightGray},
keywordstyle=\tt\bfseries,  %\color{TUGreen}\bfseries,
commentstyle=\color{DarkGray},
stringstyle=\color{red},
showstringspaces=false,
basicstyle=\small\color{black},
numbers=left,
captionpos=b,
tabsize=4,
breaklines=true
}

% -------------------------------------------------------------------
% Algorithmen
\usepackage[plain,chapter]{algorithm}
\usepackage{algorithmic}

\usepackage{enumerate}

% -------------------------------------------------------------------
% Algorithmen anpassen
\renewcommand{\algorithmicrequire}{\textit{Eingabe:}}
\renewcommand{\algorithmicensure}{\textit{Ausgabe:}}
\floatname{algorithm}{Algorithmus}
\renewcommand{\listalgorithmname}{Algorithmenverzeichnis}
\renewcommand{\algorithmiccomment}[1]{\color{grau}{// #1}}

% -------------------------------------------------------------------
% Theorem-Umgebungen
\usepackage[amsmath,thmmarks]{ntheorem}
\theoremseparator{.}
\theoremstyle{change}
\newtheorem{theorem}{Theorem}[section]
\newtheorem{satz}[theorem]{Satz}
\newtheorem{lemma}[theorem]{Lemma}
\newtheorem{korollar}[theorem]{Korollar}
\newtheorem{proposition}[theorem]{Proposition}
% Ohne Numerierung
\theoremstyle{nonumberplain}
\renewtheorem{theorem*}{Theorem}
\renewtheorem{satz*}{Satz}
\renewtheorem{lemma*}{Lemma}
\renewtheorem{korollar*}{Korollar}
\renewtheorem{proposition*}{Proposition}
% Definitionen mit \upshape
\theorembodyfont{\upshape}
\theoremstyle{change}
\newtheorem{definition}[theorem]{Definition}
\theoremstyle{nonumberplain}
\renewtheorem{definition*}{Definition}
% Kursive Schrift
\theoremheaderfont{\itshape}
\newtheorem{notation}{Notation}
\newtheorem{konvention}{Konvention}
\newtheorem{bezeichnung}{Bezeichnung}
\theoremsymbol{\ensuremath{\Box}}
\newtheorem{beweis}{Beweis}
\theoremsymbol{}
\theoremstyle{change}
\theoremheaderfont{\bfseries}
\newtheorem{bemerkung}[theorem]{Bemerkung}
\newtheorem{beobachtung}[theorem]{Beobachtung}
\newtheorem{beispiel}[theorem]{Beispiel}
\newtheorem{problem}{Problem}
\theoremstyle{nonumberplain}
\renewtheorem{bemerkung*}{Bemerkung}
\renewtheorem{beispiel*}{Beispiel}
\renewtheorem{problem*}{Problem}

% Algorithmen anpassen %
\renewcommand{\algorithmicrequire}{\textit{Eingabe:}}
\renewcommand{\algorithmicensure}{\textit{Ausgabe:}}
\floatname{algorithm}{Algorithmus}
\renewcommand{\listalgorithmname}{Algorithmenverzeichnis}
\renewcommand{\algorithmiccomment}[1]{\color{grau}{// #1}}

% Zeilenabstand einstellen %
\renewcommand{\baselinestretch}{1.25}
% Floating-Umgebungen anpassen %
\renewcommand{\topfraction}{0.9}
\renewcommand{\bottomfraction}{0.8}
% Abkuerzungen richtig formatieren %
\usepackage{xspace}
\newcommand{\vgl}{vgl.\@\xspace} 
\newcommand{\zB}{z.\nolinebreak[4]\hspace{0.125em}\nolinebreak[4]B.\@\xspace}
\newcommand{\bzw}{bzw.\@\xspace}
\newcommand{\dahe}{d.\nolinebreak[4]\hspace{0.125em}h.\nolinebreak[4]\@\xspace}
\newcommand{\etc}{etc.\@\xspace}
\newcommand{\evtl}{evtl.\@\xspace}
\newcommand{\ggf}{ggf.\@\xspace}
\newcommand{\bzgl}{bzgl.\@\xspace}
\newcommand{\so}{s.\nolinebreak[4]\hspace{0.125em}\nolinebreak[4]o.\@\xspace}
\newcommand{\iA}{i.\nolinebreak[4]\hspace{0.125em}\nolinebreak[4]A.\@\xspace}
\newcommand{\sa}{s.\nolinebreak[4]\hspace{0.125em}\nolinebreak[4]a.\@\xspace}
\newcommand{\su}{s.\nolinebreak[4]\hspace{0.125em}\nolinebreak[4]u.\@\xspace}
\newcommand{\ua}{u.\nolinebreak[4]\hspace{0.125em}\nolinebreak[4]a.\@\xspace}
\newcommand{\og}{o.\nolinebreak[4]\hspace{0.125em}\nolinebreak[4]g.\@\xspace}
\newcommand{\oBdA}{o.\nolinebreak[4]\hspace{0.125em}\nolinebreak[4]B.\nolinebreak[4]\hspace{0.125em}d.\nolinebreak[4]\hspace{0.125em}A.\@\xspace}
\newcommand{\OBdA}{O.\nolinebreak[4]\hspace{0.125em}\nolinebreak[4]B.\nolinebreak[4]\hspace{0.125em}d.\nolinebreak[4]\hspace{0.125em}A.\@\xspace}

% Leere Seite ohne Seitennummer, naechste Seite rechts
\newcommand{\blankpage}{
 \clearpage{\pagestyle{empty}\cleardoublepage}
}
