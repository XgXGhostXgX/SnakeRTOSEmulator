%========================================================================================
% TU Dortmund, Informatik Lehrstuhl VII
%========================================================================================
\chapter{Spiel}
\label{Spiel}
%
In diesem Abschnitt wird das Snake Spiel beschrieben und die Funktionen von Elementen im Gameplay oder Buttons erkl{\"a}rt. Dabei gehen wir auf die Funktionsweise des Spieles ein, die Implementierung folgt im Kapitel \ref{Spiel_Logik} : \nameref{Spiel_Logik}


\section{Spiel - Hauptmen{\"u}}
\label{Spiel_-_Hauptmenü}
%
\begin{figure}[h]
 \centering
 \includegraphics[scale=0.5]{bilder/Hauptmenü}
 \caption{Hauptmen{\"u}}
 \label{fig:hauptmenü}
\end{figure}
Das Hauptmen{\"u} (Bild \ref{fig:hauptmenü}) erscheint beim starten des Spiels und hat f{\"u}nf Buttons. Liegt der Mauszeiger {\"u}ber dem Fragezeichen-Button wird eine Anleitung des Spiels angezeigt. Dort wird die Steuerung f{\"u}r die jeweiligen Spieler angezeigt und spezielle Elemente aus dem Mehrspielermodus erkl{\"a}rt.\\
\begin{minipage}[X]{1.1\textwidth}
 \centering
 \includegraphics[scale=0.5]{bilder/Einstellungen}
 \captionof{figure}{Einstellungen}
 \label{fig:einstellungen}
\end{minipage}
\newline \\ \\
Wird auf das Zahnrad rechts oben geklickt, {\"o}ffnet sich das Einstellungsmen{\"u} (Bild \ref{fig:einstellungen}), indem verschiedene Levels ausgew{\"a}hlt werden k{\"o}nnen und die Au{\ss}enwand aktiviert und deaktiviert werden kann. Ist die Au{\ss}enwand deaktiviert, kann die Schlange beim kollidieren mit der Au{\ss}enwand nicht sterben, sondern kommt aus der gegenüberliegenden Wand wieder heraus. Durch klicken auf das Haus links oben kehrt der Spieler zum Hauptmen{\"u} zur{\"u}ck.\newline \newline \\ 
\begin{minipage}[X]{1.1\textwidth}
 \centering
 \includegraphics[scale=0.5]{bilder/Highscore}
 \captionof{figure}{Highscore}
 \label{fig:highscore}
\end{minipage}
\newline \\ \\
Wird im Hauptmen{\"u} auf den Highscore-Button geklickt, wird der Highscore (Bild \ref{fig:highscore}) aus den bisher gespielten Spielen angezeigt. Dabei sind die Highscores prim{\"a}r nach Punkten und sekund{\"a}r nach Zeit sortiert. Au{\ss}erdem steht in jeweils der letzten Spalte der Highscores das Level, indem gespielt wurde. Darüber hinaus kann die Sortierung durch Klicken auf das Punkte-Label umgekehrt werden. Mit Klicken auf das Haus links oben kehrt der Spieler zur{\"u}ck zum Hauptmen{\"u}.
\\
 Klickt der Spieler auf den \glqq One Player Game\grqq{}-Button {\"o}ffnet sich ein Men{\"u}, indem der Spieler seinen Namen eintragen kann. Dies kann durch Klicken auf die jeweiligen Buttons auf dem Bildschirm oder durch Eingabe {\"u}ber die Tastatur geschehen. Das Spiel wird dann gestartet, wenn der StartGame-Button bet{\"a}tigt wurde.\\
W{\"a}hlt der Spieler den Mehrspielermodus, durch Bet{\"a}tigen des 	\glqq Two Player Game\grqq{}-Button im Hauptmen{\"u}, {\"o}ffnet sich wieder ein Men{\"u} zum Eintragen der Spielernamen und das Spiel kann durch das Bet{\"a}tigen des StartGame-Buttons gestartet werden.  


\section{Spiel - Einzelspielermodus}
\label{Spiel_-_Einzelspielermodus}
%
\begin{minipage}[X]{1.1\textwidth}
 \centering
 \includegraphics[scale=0.5]{bilder/Einzelspielermodus}
 \captionof{figure}{Einzelspieler Spiel}
 \label{fig:einzelspielermodus}
\end{minipage}
	Der Einzelspielermodus ist im Grunde das traditionelle Snake Game. Die Schlange l{\"a}sst sich steuern durch Bet{\"a}tigen der Tasten \glqq W\grqq{}, \glqq  A\grqq{}, \glqq S\grqq{}, \glqq D\grqq{} und bewegt sich zu Anfang alle f{\"u}nf Frames. Die gr{\"u}nen Elemente sind das Essen der Schlange, welches die Schlange wachsen l{\"a}sst. Dabei w{\"a}chst die Schlange um eine Gr{\"o}{\ss}e beim Verzehren des kleinen Food-Elements und um zwei beim Verzehren des gro{\ss}en Superfood-Elements. Alle 25 Punkte wird die Schlange schneller, welches die Schwierigkeit des Spiels erhöht.
	Das Spiel kann durch die entsprechenden Levels erschwert werden. Level f{\"u}nf hat ein lilafarbenes Teleport-Element, welches die Schlange zu einem anderen Teleport-Element teleportiert. Dabei teleportieren die oberen Elemente zum n{\"a}chsten rechten Element und die unteren Elemente zum n{\"a}chsten linken Element. Level 6 ver{\"a}ndert sein inneres Wandmuster nach dem Verzehren des Food-Elements, jedoch nicht nach Verzehren des Superfood-Elements. Das Spiel endet, wenn die Schlange mit der Wand oder mit sich selber kollidiert.   

\newpage
\section{Spiel - Mehrspielermodus}
\label{Spiel_-_Mehrspielermodus}
%
\textcolor{white}{easily}
\newline 
\begin{minipage}[X]{1.0\textwidth}
 \centering
 \includegraphics[scale=0.5]{bilder/Mehrspielermodus}
 \captionof{figure}{Mehrspieler Spiel}
 \label{fig:mehrspielermodus}
\end{minipage}
\\
	Beim Mehrspielermodus spielen zwei Spieler gegeneinander und versuchen das Spiel durch t{\"o}ten der Schlange des Gegenspielers zu gewinnen. Spieler eins bewegt die rote Schlange mit den Tasten \glqq W\grqq{}, \glqq  A\grqq{}, \glqq S\grqq{}, \glqq D\grqq{} und Spieler zwei bewegt die blaue Schlange mit den Pfeiltasten. Im Mehrspielermodus gibt es au{\ss}erdem Spezial-Elemente, wie das cyanfarbene Freeze-Element, das gelbe Reduce-Element und das graue Inverse Element. Diese Elemente k{\"o}nnen aufgesammelt werden und mit Leertaste f{\"u}r Spieler eins und mit Shift f{\"u}r Spieler zwei eingesetzt werden. Dabei kann nur ein Element gleichzeitig in der Tasche eines jeweiligen Spielers sein. Beim Einsatz des Freeze-Elementes kann die gegnerische Schlange sich f{\"u}r 15 Schritte nicht bewegen. Beim Einsatz des Reduce-Elementes verringert sich die L{\"a}nge der gegnerischen Schlange um eine Gr{\"o}{\ss}e. Das Inverse-Element invertiert die Steuerung des Gegenspielers f{\"u}r 50 Schritte. Das Spiel ist beendet, wenn eines der Spieler verliert. Kollidiert die Schlange eines Spielers mit der Wand, sich selber oder dem K{\"o}rper der gegnerischen Schlange, verliert derjenige Spieler. Kollidieren beide Schlangen Kopf an Kopf gewinnt derjenige Spieler mit der gr{\"o}{\ss}eren Schlange.  

\section{Spiel - Spielende}
\label{Spiel_-_Spielende}
%
\textcolor{white}{easily}
\newline 
\begin{minipage}[X]{1.1\textwidth}
 \centering
 \includegraphics[scale=0.5]{bilder/Spielende}
 \captionof{figure}{Spielende}
 \label{fig:spielende}
\end{minipage}
\\ \\ 
	Ist das Spiel zu Ende erscheint eine kleine Animation und mehrere Buttons. Der Restart-Button startet das Spiel neu und der Highscore-Button zeigt den Highscore an. Au{\ss}erdem steht {\"u}ber dem Spielfeld welcher Spieler gewonnen bzw. verloren hat.  

%
