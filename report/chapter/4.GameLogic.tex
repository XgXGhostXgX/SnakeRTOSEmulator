%========================================================================================
% TU Dortmund, Informatik Lehrstuhl VII
%========================================================================================

\chapter{Spiel Logik}
\label{Spiel_Logik}
%
In diesem Kapitel gehen wir auf die Hauptfunktion des eigentlichen Spielablaufs, $vGameScreen()$, und der daf{\"u}r ben{\"o}tigten einzelnen Funktionen ein. Wie in den meisten Computerspielen besteht die Hauptfunkion im groben aus einer einzigen gro{\ss}en Schleife, in der zum der aktuelle Spielzustand abh{\"a}ngig von den Eingaben der Spieler ge{\"a}ndert wird, und zum anderen der aktuelle Spielzustand grafisch aufgearbeitet und auf den Bildschirm gebracht wird. Vor dem Start dieser Schleife findet noch eine Initialisierung statt, welche den Spielzustand auf den Anfang des Spiels setzt.
%

\section{Initialisierung}
\label{Initialisierung}
%
Das erste, was nach Starten des Spiel geschieht, ist das Erstellen eine Art $seed$ f{\"u}r die $rand()$ Funktion der C - Programmiersprache. Da die $rand()$ Funktion die verstrichene Zeit seit Aufruf der Funktion in der sie aufgerufen wird als Basis f{\"u}r die Berechnung der Pseudozufallszahl nimmt, die ersten Aufrufe der $rand()$ Funktion immer die gleichen Werte zur{\"u}ckgeben, und somit die Schlangen der Spieler sowie das erste $foodElement$ immer an der gleichen Stelle starten. Um den entgegenzuwirken berechnet die Funktion als erstes die letzten beiden Ziffern des Produkts der x- und y-Koordinaten der Mausposition beim klicken des Start-Buttons und dekrementiert diese Zahl bis zur null. Durch diese minimale verstrichene Zeitspanne {\"a}ndern sich die Ergebnisse der ersten $rand()$-Aufrufe und damit der Startzustand des Spielfelds, vorrausgesetzt man dr{\"u}ckt beim Start nicht auf den exakt gleichen Pixel wie vorher.
Anschlie{\ss}end werden, je nach gew{\"a}hltem Level, die W{\"a}nde sowie alle weiteren festen Elemente des Levels an ihre entsprechenden Stellen ins $fieldArray$ geschrieben.
Daraufhin werden die Variablen f{\"u}r die Position der beiden Spieler, $p1X, p1Y, p2X$ und $p2Y$, sowie das erste $foodElement$, auf zuf{\"a}llige Positionen im Spielfeld gesetzt und die zugeh{\"o}rigen verketteten Listen f{\"u}r die Schlangen mit bereits 2 weiteren "Schlangenteilen" initialisiert. Dies geschieht auch im Singleplayer-Modus f{\"u}r die Schlange von Spieler 2, da der Code ansonsten aufgrund einer m{\"o}glicherweise nicht initialisierten Spieler-2-Schlange nicht kompiliert. Einzig der Eintrag im $fieldArray()$  wird im Singleplayer-Modus ausgelassen, da offensichtlich kein Spieler 2 existiert.
Nun ist das Spielfeld soweit initialisiert dass das eigentliche Spiel beginnen kann.
%

\section{Gameloop}
\label{Gameloop}
%

%


   

%
