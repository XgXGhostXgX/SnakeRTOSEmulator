%========================================================================================
% TU Dortmund, Informatik Lehrstuhl VII
%========================================================================================

\chapter{Einleitung}
\label{Einleitung}

Es wird hier kurz und knapp erz{\"a}hlt wie das Projekt zustande gekommen ist und in welcher Entwicklungsumgebung gearbeitet wurde.

\section{Motivation und Hintergrund}
\label{Motivation_und_Hintergrund}
%
Durch die Vergabe des Projektes war uns klar ein kleineres Retro Spiel zu programmieren, dabei kam Snake schnell als Idee. Jedoch war unser Ziel dar{\"u}ber hinaus als nur das klassische Spiel zu implementieren. Es kamem sehr schnell viele Erweiterungsvorschl{\"a}ge wie mehr und variierte Food-Elemente oder spezielle Level oder Mehrspielermodi. Inbesondere gab es sogar noch weitere Level Ideen und weitere Spielmodi, die aber wegen zeitlichen begrenzen des Projekts vernachl{\"a}ssigt wurden und daf{\"u}r gesorgt wurde, dass die Erweiterungen reibungslos funktionieren.


\section{Aufbau und Umgebung der Arbeit}
\label{Aufbau_und_Umgebung_der_Arbeit}
%
Die Vorgabe ein Echtzeitbetriebssystem (engl. RTOS) an der Universit{\"a}t zu nutzen, war wegen der Corona-Pandemie eine unm{\"o}gliche Aufgabe. Dadurch wurde alles online verschoben, dadurch war ein Emulator, welcher solch ein Echtzeitbetriebssystem simuliert, das Grundwerkzeug und Grundumgebung, womit das Projekt gestartet hat. Durch das einarbeiten in den Code des Emulators wurde es mit der Zeit immer verst{\"a}ndlicher den Aufbau einfacher das Spiel zu implementieren. Unter anderem hat der Web Cloud GitHub die Zusammenarbeit von uns sehr vereinfacht. Dadurch konnte strukturiert und effektiv an dem Projekt gearbeitet werden. Durch den WebClient BigBlueButton konnten wir uns digital treffen und Problemen oder Ideen als Gruppe besprechen. 