%========================================================================================
% TU Dortmund, Informatik Lehrstuhl VII
%========================================================================================

\chapter{Einleitung}
\label{Einleitung}

Hier gehen wir kurz darauf ein, wie das Projekt zustande gekommen ist und in welcher Entwicklungsumgebung gearbeitet wurde.

\section{Motivation und Hintergrund}
\label{Motivation_und_Hintergrund}
%
Durch die Vergabe des Projektes war uns klar ein kleineres Retro Spiel zu programmieren, nach einiger Überlegung und Diskussion zu einfachen aber leicht erweiterbaren Spielen kamen wir auf das Snake. Jedoch war unser Ziel dar{\"u}ber hinaus mehr als nur das klassische Spiel zu implementieren. Es kamen sehr schnell viele Erweiterungsvorschl{\"a}ge wie mehr und variierte Food-Elemente oder spezielle Level oder Mehrspielermodi. Insbesondere gab es sogar noch weitere Ideen zu neuen Leveln und weiteren Spielmodi, die aber wegen zeitlichen Begrenzungen des Projekts vernachl{\"a}ssigt wurden und daf{\"u}r mehr dafür gesorgt wurde, dass die implementierten Erweiterungen reibungslos funktionieren.


\section{Aufbau und Umgebung der Arbeit}
\label{Aufbau_und_Umgebung_der_Arbeit}
%
Die Vorgabe ein Echtzeitbetriebssystem (engl. RTOS) an der Universit{\"a}t im Labor zu nutzen, war wegen der Corona-Pandemie eine unm{\"o}gliche Aufgabe. Auf Grund dessen wurde das gesamte Fachprojekt online abgehalten. Dadurch war ein Emulator, welcher solch ein Echtzeitbetriebssystem simuliert, das Grundwerkzeug und Grundumgebung, womit das Projekt realisiert wurde. Durch das Einarbeiten in den Code des Emulators wurde es mit der Zeit immer verst{\"a}ndlicher den Aufbau des Simulators zu verstehen um das Spiel zu implementieren. Unter anderem hat die Versionsverwaltung mit GitHub die Zusammenarbeit von uns sehr vereinfacht. Dadurch konnte strukturiert und effektiv an dem Projekt gearbeitet werden. Durch den WebClient BigBlueButton konnten wir digitale Treffen abhalten, in denen Probleme und Ideen als Gruppe besprochen wurden. 